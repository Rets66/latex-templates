% settiong doc type
\documentclass[10pt, a4j]{jsarticle}
%
%
% table of contents fix
\usepackage{atbegshi}
\ifnum 42146=\euc"A4A2
	\AtBeginShipoutFirst{\special{pdf:tounicode EUC-UCS2}}
\else
	\AtBeginShipoutFirst{\special{pdf:tounicode 90ms-RKSJ-UCS2}}
\fi
%
%
% import packages
\usepackage{amsmath}
\usepackage{amssymb}
\usepackage{mathrsfs}
\usepackage{bm}
\usepackage{ntheorem}
\usepackage{comment}
\usepackage{ascmac}				% \begin{itembox | screen | boxnote | shadebox}
\usepackage{float}				% [htbp]
\usepackage{subfigure}
\usepackage[dvipdfmx]{hyperref}
\usepackage[dvipdfmx]{graphicx}	% extractbb example.{png | jpg | ...etc.}
\usepackage{enumitem}			% \begin{enumerate}[(1)] \item ... \end{enumetate}
%
%
% layout settings
\setlength{\textwidth}{160truemm}      % テキスト幅: 210-(25+25)=160mm
\setlength{\fullwidth}{\textwidth}     % ページ全体の幅
\setlength{\oddsidemargin}{25truemm}   % 左余白
\addtolength{\oddsidemargin}{-1truein} % 左位置デフォルトから-1inch
%
\setlength{\topmargin}{25truemm}       % 上余白
\setlength{\textheight}{242truemm}     % テキスト高さ: 297-(25+30)=242mm
\addtolength{\topmargin}{-1.5truein}   % 上位置デフォルトから-1.5inch
%
%
% lines par page
\def\linesparpage#1{\baselineskip=\textheight
   \divide\baselineskip by #1}
%
%
% subsubsubsection
\makeatletter
\newcommand{\subsubsubsection}{\@startsection{paragraph}{4}{\z@}%
	{1.0\Cvs \@plus.5\Cdp \@minus.2\Cdp}%
	{.1\Cvs \@plus.3\Cdp}%
	{\reset@font\sffamily\normalsize}
}
%
%
% theorem
\newtheorem{Definition}{定義}[section]
\newtheorem{Theorem}{定理}[section]
\newtheorem{Lemma}{補題}[section]
\newtheorem{Proof}{証明}[section]
%
% Contents depth
\setcounter{tocdepth}{2}
%
% Page style
\pagestyle{empty}
%
% bibliography style setting
%\bibliographystyle{jplain}
%\bibliographystyle{junsrt}
%
%
% listings settings
\begin{comment}
\usepackage{here}
\usepackage{txfonts}
\usepackage{listings, jlisting}
\renewcommand{\lstlistingname}{リスト}
\lstset{
	language=c,
	basicstyle=\ttfamily\scriptsize,
	commentstyle=\textit,
	classoffset=1,
	keywordstyle=\bfseries,
	frame=tRBl,
	framesep=5pt,
	showstringspaces=false,
	numbers=left,
	stepnumber=1,
	numberstyle=\tiny,
	tabsize=2
}
\end{comment}
% 図と図の間の余白
%\setlength\floatsep{0truemm}
% 本文と図の間の余白
%\setlength\textfloatsep{0truemm}
% 本文中の図の余白
%\setlength\intextsep{0truemm}
% 図とキャプションの間の余白
%\setlength\abovecaptionskip{0truemm}
%
% title
\title{タイトル}
\author{1133033904 白石 裕輝\footnote{岐阜大学 工学部 電気電子・情報工学科 情報コース3年}}
\date{\today}
%
% ----------------------------------------------------------------
\begin{document}
%\linesparpage{50}
%
% title page
\maketitle
%
% 本文
%
\end{document}
% ----------------------------------------------------------------
%
\begin{comment}
% ----------------------------------------------------------------
% Tips
% ----------------------------------------------------------------

% equation
\[  \]
% or
\begin{align}
	
\end{align}


% insert figure
\begin{figure}
	\centering
	\includegraphics[width=80mm]{./figs/example.eps}
	\caption{caption}
	\label{fig:example}
\end{figure}


% insert table (for picture)
\begin{table}
	\caption{caption}
	\label{table:example}
	\centering
	\includegraphics[width=100mm]{./tables/example.eps}
\end{table}


% insert table
\begin{table}
	\caption{caption}
	\label{table:example}
	\begin{tabular}{|l|c|r|r|}
		\hline header1 & header2 & header3 \\ \hline \hline
		cell1.1 & cell1.2 & cell1.3 \\ \hline
		cell2.1 & cell2.2 & cell2.3 \\ \hline
	\end{tabular}
\end{table}


% subfigure
\begin{figure}[htbp]
	\subfigure[caption1]{
		\includegraphics[width=80mm]{./figs/example1.eps}
		\label{fig:example1}
	}
	\subfigure[caption2]{
		\includegraphics[width=80mm]{./figs/example2.eps}
		\label{fig:example2}
	}
	\caption{caption}
	\label{fig:example}
\end{figure}


% (1), (2), (3)...
\begin{enumerate}[label=(\arabic*)]
	\item 
	\item 
	\item 
\end{enumerate}


% definition and description
\begin{description}
	\item[Earth]~\\
	The Earth is the third planet from the Sun and it is the only planet known to have life on it. ...
\end{description}


% insert list
\lstinputlisting[language=bash, frame=tlrb, caption={example.sh}, label={list:example_sh}]{./src/example.sh}


% <code> ... </code> 
\begin{verbatim}
int a %= 4;
\end{verbatim}

\verb|int a %=Abstract
% ----------------------------------------------------------------
% hamiltonian
\mathscr{H}


% URL
\url{http://www.example.com}
\href{http://www.example.com}{display text}


% text styles
\textbf{...}, \emph{...}, \textit{...}, \textmd{...}, \textrm{...}, \textsc{...}, \textsf{...}, \textsl{...}, \texttt{...}, \textup{...}
\end{comment}
