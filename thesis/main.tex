% ----------------------------------------------------------------
% Document class
% ----------------------------------------------------------------
\documentclass[a4paper, dvipdfmx, oneside, 12pt]{jsbook}

% ----------------------------------------------------------------
% Import packages
% ----------------------------------------------------------------
\usepackage{amsmath, amssymb}         % 数式,シンボル
% \usepackage{newtxtext, newtxmath}   % 新 TX フォント
\usepackage{latexsym}                 % シンボル
\usepackage[amsmath]{ntheorem}        % 定理
\usepackage{bm}                       % 太字ベクトル
% \usepackage{algorithm}              % アルゴリズム
% \usepackage[noend]{algpseudocode}   % 疑似コード
\usepackage{booktabs}                 % 罫線
\usepackage{float}                    % 図表の配置
\usepackage[hiresbb]{graphicx}        % 画像の挿入
\usepackage{hyperref}                 % ハイパーリンク
\usepackage{subcaption}               % サブキャプション
\usepackage{enumitem}                 % 箇条書きの書式変更
\usepackage{array}                    % 表の罫線が繋がらない問題対策
\usepackage{pxjahyper}                % hyperref の文字化け対策

% ----------------------------------------------------------------
% Page settings
% ----------------------------------------------------------------
% Margin
\usepackage[
  driver  = dvipdfm,
  truedimen,
  top     = 30truemm,
  bottom  = 25truemm,
  left    = 20truemm,
  right   = 20truemm
]{geometry}

% \setlength\floatsep{0truemm}           % 図と図の間の余白
% \setlength\textfloatsep{0truemm}       % 本文と図の間の余白
% \setlength\intextsep{0truemm}          % 本文中の図の余白
% \setlength\abovecaptionskip{0truemm}   % 図とキャプションの間の余白

% Page number style
\makeatletter
\def\ps@plainfoot{%
  \let\@mkboth\@gobbletwo
  \let\@oddhead\@empty
  \def\@oddfoot{\normalfont\hfil-- \thepage\ --\hfil}%
  \let\@evenhead\@empty
  \let\@evenfoot\@oddfoot}
\let\ps@plain\ps@plainfoot
\makeatother
\pagestyle{plain}

\setlength\footskip{2\baselineskip}
\addtolength{\textheight}{-2\baselineskip}

% Contents depth
\setcounter{tocdepth}{3}

% Caption style
\captionsetup{
  format      = plain,
  labelformat = simple,
  labelsep    = quad,
  font        = normalsize
}

% Bibliography style
% \bibliographystyle{plain}

% Section font
\renewcommand{\headfont}{\bfseries}

% Algorithm w/o ``do''
% \algrenewcommand\algorithmicdo{}

% ----------------------------------------------------------------
% Commands
% ----------------------------------------------------------------
% \begin{theorem}
\theorembodyfont{\normalfont}
\newtheorem{definition}{定義}[section]
\newtheorem{theorem}{定理}[section]
\newtheorem{lemma}{補題}[section]
\newtheorem{proposition}{命題}[section]
\newtheorem{axiom}{公理}[section]
\newtheorem{corollary}{系}[section]
\newtheorem{proof}{証明}[section]

% \qed
\def\qed{\hfill $\Box$}

% \argmax, \argmin
\newcommand{\argmax}{\operatornamewithlimits{\mathrm{arg\,max}}}
\newcommand{\argmin}{\operatornamewithlimits{\mathrm{arg\,min}}}

% \ForTo
% \algnewcommand\algorithmicto{\textbf{to}}
% \algdef{SE}[FOR]{ForTo}{EndFor}[2]{\algorithmicfor\ #1\ \algorithmicto\ #2\ \algorithmicdo}{\algorithmicend\ \algorithmicfor}%
% algtext*{EndFor}

% ----------------------------------------------------------------
% Document
% ----------------------------------------------------------------
\begin{document}
  % Title
  % title
\begin{titlepage}
  \title{
    Title
  }
  \author{
    Yuki \textsc{Shiraishi}\\
    {\small Department of Electrical, Electronics and Computer Engineering} \\
    {\small Gifu University}
  }
  \date{March 17, 2017}
\end{titlepage}

  \maketitle

  % Front matter
  \frontmatter
  \chapter{Abstract}
Lorem ipsum dolor sit amet, consectetur adipiscing elit, sed do eiusmod tempor
incididunt ut labore et dolore magna aliqua. Ut enim ad minim veniam, quis
nostrud exercitation ullamco laboris nisi ut aliquip ex ea commodo consequat.
Duis aute irure dolor in reprehenderit in voluptate velit esse cillum dolore eu
fugiat nulla pariatur. Excepteur sint occaecat cupidatat non proident, sunt in
culpa qui officia deserunt mollit anim id est laborum.

Lorem ipsum dolor sit amet, consectetur adipiscing elit, sed do eiusmod tempor
incididunt ut labore et dolore magna aliqua. Ut enim ad minim veniam, quis
nostrud exercitation ullamco laboris nisi ut aliquip ex ea commodo consequat.
Duis aute irure dolor in reprehenderit in voluptate velit esse cillum dolore eu
fugiat nulla pariatur. Excepteur sint occaecat cupidatat non proident, sunt in
culpa qui officia deserunt mollit anim id est laborum.

  \tableofcontents

  % Main matter
  \mainmatter
  \chapter{はじめに}
私は今日初めてこの学習院というものの中に這入はいりました。もっとも以前から学習院は多分この見当
だろうぐらいに考えていたには相違そういありませんが、はっきりとは存じませんでした。中へ這入った
のは無論今日が初めてでございます。
さきほど岡田さんが紹介しょうかいかたがたちょっとお話になった通りこの春何か講演をというご注文で
ありましたが、その当時は何か差支さしつかえがあって、――岡田さんの方が当人の私よりよくご記憶と
見えてあなたがたにご納得のできるようにただいまご説明がありましたが、とにかくひとまずお断りを
致いたさなければならん事になりました。しかしただお断りを致すのもあまり失礼と存じまして、
この次には参りますからという条件をつけ加えておきました。その時念のためこの次はいつごろに
なりますかと岡田さんに伺うかがいましたら、此年ことしの十月だというお返事であったので、
心のうちに春から十月までの日数を大体繰くってみて、それだけの時間があればそのうちにどうにか
できるだろうと思ったものですから、よろしゅうございますとはっきりお受合うけあい申したので
あります。ところが幸か不幸か病気に罹かかりまして、九月いっぱい床とこについておりますうちに
お約束やくそくの十月が参りました。十月にはもう臥ふせってはおりませんでしたけれども、何しろ
ひょろひょろするので講演はちょっとむずかしかったのです。しかしお約束を忘れてはならないの
ですから、腹の中では、今に何か云いって来られるだろう来られるだろうと思って、内々ないないは
怖こわがっていました。

そのうちひょろひょろもついに癒なおってしまったけれども、こちらからは十月末まで何の
ご沙汰さたもなく打ち過ぎました。私は無論病気の事をご通知はしておきませんでしたが、二三の新聞に
ちょっと出たという話ですから、あるいはその辺の事情を察せられて、誰だれかが私の代りに講演を
やって下さったのだろうと推測して安心し出しました。ところへまた岡田さんがまた突然とつぜん
見えたのであります。岡田さんはわざわざ長靴を穿はいて見えたのであります。
(もっとも雨の降る日であったからでもありましょうが、)そう云った身拵みごしらえで、
早稲田わせだの奥おくまで来て下すって、例の講演は十一月の末まで繰り延ばす事にしたから
約束通りやってもらいたいというご口上なのです。私はもう責任を逃のがれたように考えていたもの
ですから実は少々驚おどろきました。しかしまだ一カ月も余裕よゆうがあるから、その間にどうか
なるだろうと思って、よろしゅうございますとまたご返事を致しました。


\section{学習院}
私は今日初めてこの学習院というものの中に這入はいりました。もっとも以前から学習院は多分この見当
だろうぐらいに考えていたには相違そういありませんが、はっきりとは存じませんでした。中へ這入った
のは無論今日が初めてでございます。

\subsection{十月}
さきほど岡田さんが紹介しょうかいかたがたちょっとお話になった通りこの春何か講演をというご注文で
ありましたが、その当時は何か差支さしつかえがあって、――岡田さんの方が当人の私よりよくご記憶と
見えてあなたがたにご納得のできるようにただいまご説明がありましたが、とにかくひとまずお断りを
致いたさなければならん事になりました。しかしただお断りを致すのもあまり失礼と存じまして、
この次には参りますからという条件をつけ加えておきました。その時念のためこの次はいつごろに
なりますかと岡田さんに伺うかがいましたら、此年ことしの十月だというお返事であったので、
心のうちに春から十月までの日数を大体繰くってみて、それだけの時間があればそのうちにどうにか
できるだろうと思ったものですから、よろしゅうございますとはっきりお受合うけあい申したので
あります。ところが幸か不幸か病気に罹かかりまして、九月いっぱい床とこについておりますうちに
お約束やくそくの十月が参りました。十月にはもう臥ふせってはおりませんでしたけれども、何しろ
ひょろひょろするので講演はちょっとむずかしかったのです。しかしお約束を忘れてはならないの
ですから、腹の中では、今に何か云いって来られるだろう来られるだろうと思って、内々ないないは
怖こわがっていました。

\subsection{十一月}
そのうちひょろひょろもついに癒なおってしまったけれども、こちらからは十月末まで何の
ご沙汰さたもなく打ち過ぎました。私は無論病気の事をご通知はしておきませんでしたが、二三の新聞に
ちょっと出たという話ですから、あるいはその辺の事情を察せられて、誰だれかが私の代りに講演を
やって下さったのだろうと推測して安心し出しました。ところへまた岡田さんがまた突然とつぜん
見えたのであります。岡田さんはわざわざ長靴を穿はいて見えたのであります。

(もっとも雨の降る日であったからでもありましょうが、)そう云った身拵みごしらえで、
早稲田わせだの奥おくまで来て下すって、例の講演は十一月の末まで繰り延ばす事にしたから
約束通りやってもらいたいというご口上なのです。私はもう責任を逃のがれたように考えていたもの
ですから実は少々驚おどろきました。しかしまだ一カ月も余裕よゆうがあるから、その間にどうか
なるだろうと思って、よろしゅうございますとまたご返事を致しました。
私は今日初めてこの学習院というものの中に這入はいりました。もっとも以前から学習院は多分この見当
だろうぐらいに考えていたには相違そういありませんが、はっきりとは存じませんでした。中へ這入った
のは無論今日が初めてでございます。


\section{岡田さん}
私は今日初めてこの学習院というものの中に這入はいりました。もっとも以前から学習院は多分この見当
だろうぐらいに考えていたには相違そういありませんが、はっきりとは存じませんでした。中へ這入った
のは無論今日が初めてでございます。
さきほど岡田さんが紹介しょうかいかたがたちょっとお話になった通りこの春何か講演をというご注文で
ありましたが、その当時は何か差支さしつかえがあって、――岡田さんの方が当人の私よりよくご記憶と
見えてあなたがたにご納得のできるようにただいまご説明がありましたが、とにかくひとまずお断りを
致いたさなければならん事になりました。しかしただお断りを致すのもあまり失礼と存じまして、
この次には参りますからという条件をつけ加えておきました。その時念のためこの次はいつごろに
なりますかと岡田さんに伺うかがいましたら、此年ことしの十月だというお返事であったので、
心のうちに春から十月までの日数を大体繰くってみて、それだけの時間があればそのうちにどうにか
できるだろうと思ったものですから、よろしゅうございますとはっきりお受合うけあい申したので
あります。ところが幸か不幸か病気に罹かかりまして、九月いっぱい床とこについておりますうちに
お約束やくそくの十月が参りました。十月にはもう臥ふせってはおりませんでしたけれども、何しろ
ひょろひょろするので講演はちょっとむずかしかったのです。しかしお約束を忘れてはならないの
ですから、腹の中では、今に何か云いって来られるだろう来られるだろうと思って、内々ないないは
怖こわがっていました。

  \chapter{関連研究}
私は今日初めてこの学習院というものの中に這入はいりました。もっとも以前から学習院は多分この見当
だろうぐらいに考えていたには相違そういありませんが、はっきりとは存じませんでした。中へ這入った
のは無論今日が初めてでございます。
さきほど岡田さんが紹介しょうかいかたがたちょっとお話になった通りこの春何か講演をというご注文で
ありましたが、その当時は何か差支さしつかえがあって、――岡田さんの方が当人の私よりよくご記憶と
見えてあなたがたにご納得のできるようにただいまご説明がありましたが、とにかくひとまずお断りを
致いたさなければならん事になりました。しかしただお断りを致すのもあまり失礼と存じまして、
この次には参りますからという条件をつけ加えておきました。その時念のためこの次はいつごろに
なりますかと岡田さんに伺うかがいましたら、此年ことしの十月だというお返事であったので、
心のうちに春から十月までの日数を大体繰くってみて、それだけの時間があればそのうちにどうにか
できるだろうと思ったものですから、よろしゅうございますとはっきりお受合うけあい申したので
あります。ところが幸か不幸か病気に罹かかりまして、九月いっぱい床とこについておりますうちに
お約束やくそくの十月が参りました。十月にはもう臥ふせってはおりませんでしたけれども、何しろ
ひょろひょろするので講演はちょっとむずかしかったのです。しかしお約束を忘れてはならないの
ですから、腹の中では、今に何か云いって来られるだろう来られるだろうと思って、内々ないないは
怖こわがっていました。

そのうちひょろひょろもついに癒なおってしまったけれども、こちらからは十月末まで何の
ご沙汰さたもなく打ち過ぎました。私は無論病気の事をご通知はしておきませんでしたが、二三の新聞に
ちょっと出たという話ですから、あるいはその辺の事情を察せられて、誰だれかが私の代りに講演を
やって下さったのだろうと推測して安心し出しました。ところへまた岡田さんがまた突然とつぜん
見えたのであります。岡田さんはわざわざ長靴を穿はいて見えたのであります。
(もっとも雨の降る日であったからでもありましょうが、)そう云った身拵みごしらえで、
早稲田わせだの奥おくまで来て下すって、例の講演は十一月の末まで繰り延ばす事にしたから
約束通りやってもらいたいというご口上なのです。私はもう責任を逃のがれたように考えていたもの
ですから実は少々驚おどろきました。しかしまだ一カ月も余裕よゆうがあるから、その間にどうか
なるだろうと思って、よろしゅうございますとまたご返事を致しました。


\section{学習院}
私は今日初めてこの学習院というものの中に這入はいりました。もっとも以前から学習院は多分この見当
だろうぐらいに考えていたには相違そういありませんが、はっきりとは存じませんでした。中へ這入った
のは無論今日が初めてでございます。
さきほど岡田さんが紹介しょうかいかたがたちょっとお話になった通りこの春何か講演をというご注文で
ありましたが、その当時は何か差支さしつかえがあって、――岡田さんの方が当人の私よりよくご記憶と
見えてあなたがたにご納得のできるようにただいまご説明がありましたが、とにかくひとまずお断りを
致いたさなければならん事になりました。しかしただお断りを致すのもあまり失礼と存じまして、
この次には参りますからという条件をつけ加えておきました。その時念のためこの次はいつごろに
なりますかと岡田さんに伺うかがいましたら、此年ことしの十月だというお返事であったので、
心のうちに春から十月までの日数を大体繰くってみて、それだけの時間があればそのうちにどうにか
できるだろうと思ったものですから、よろしゅうございますとはっきりお受合うけあい申したので
あります。ところが幸か不幸か病気に罹かかりまして、九月いっぱい床とこについておりますうちに
お約束やくそくの十月が参りました。十月にはもう臥ふせってはおりませんでしたけれども、何しろ
ひょろひょろするので講演はちょっとむずかしかったのです。しかしお約束を忘れてはならないの
ですから、腹の中では、今に何か云いって来られるだろう来られるだろうと思って、内々ないないは
怖こわがっていました。

\subsection{CAN}
そのうちひょろひょろもついに癒なおってしまったけれども、こちらからは十月末まで何の
ご沙汰さたもなく打ち過ぎました。私は無論病気の事をご通知はしておきませんでしたが、二三の新聞に
ちょっと出たという話ですから、あるいはその辺の事情を察せられて、誰だれかが私の代りに講演を
やって下さったのだろうと推測して安心し出しました。ところへまた岡田さんがまた突然とつぜん
見えたのであります。岡田さんはわざわざ長靴を穿はいて見えたのであります。
(もっとも雨の降る日であったからでもありましょうが、)そう云った身拵みごしらえで、
早稲田わせだの奥おくまで来て下すって、例の講演は十一月の末まで繰り延ばす事にしたから
約束通りやってもらいたいというご口上なのです。私はもう責任を逃のがれたように考えていたもの
ですから実は少々驚おどろきました。しかしまだ一カ月も余裕よゆうがあるから、その間にどうか
なるだろうと思って、よろしゅうございますとまたご返事を致しました。

\subsection{Chord}
私は今日初めてこの学習院というものの中に這入はいりました。もっとも以前から学習院は多分この見当
だろうぐらいに考えていたには相違そういありませんが、はっきりとは存じませんでした。中へ這入った
のは無論今日が初めてでございます。
さきほど岡田さんが紹介しょうかいかたがたちょっとお話になった通りこの春何か講演をというご注文で
ありましたが、その当時は何か差支さしつかえがあって、――岡田さんの方が当人の私よりよくご記憶と
見えてあなたがたにご納得のできるようにただいまご説明がありましたが、とにかくひとまずお断りを
致いたさなければならん事になりました。しかしただお断りを致すのもあまり失礼と存じまして、
この次には参りますからという条件をつけ加えておきました。その時念のためこの次はいつごろに
なりますかと岡田さんに伺うかがいましたら、此年ことしの十月だというお返事であったので、
心のうちに春から十月までの日数を大体繰くってみて、それだけの時間があればそのうちにどうにか
できるだろうと思ったものですから、よろしゅうございますとはっきりお受合うけあい申したので
あります。ところが幸か不幸か病気に罹かかりまして、九月いっぱい床とこについておりますうちに
お約束やくそくの十月が参りました。十月にはもう臥ふせってはおりませんでしたけれども、何しろ
ひょろひょろするので講演はちょっとむずかしかったのです。しかしお約束を忘れてはならないの
ですから、腹の中では、今に何か云いって来られるだろう来られるだろうと思って、内々ないないは
怖こわがっていました。

\subsection{Skip Graph}
そのうちひょろひょろもついに癒なおってしまったけれども、こちらからは十月末まで何の
ご沙汰さたもなく打ち過ぎました。私は無論病気の事をご通知はしておきませんでしたが、二三の新聞に
ちょっと出たという話ですから、あるいはその辺の事情を察せられて、誰だれかが私の代りに講演を
やって下さったのだろうと推測して安心し出しました。ところへまた岡田さんがまた突然とつぜん
見えたのであります。岡田さんはわざわざ長靴を穿はいて見えたのであります。
(もっとも雨の降る日であったからでもありましょうが、)そう云った身拵みごしらえで、
早稲田わせだの奥おくまで来て下すって、例の講演は十一月の末まで繰り延ばす事にしたから
約束通りやってもらいたいというご口上なのです。私はもう責任を逃のがれたように考えていたもの
ですから実は少々驚おどろきました。しかしまだ一カ月も余裕よゆうがあるから、その間にどうか
なるだろうと思って、よろしゅうございますとまたご返事を致しました。


\section{岡田さん}
  私は今日初めてこの学習院というものの中に這入はいりました。もっとも以前から学習院は多分この見当
  だろうぐらいに考えていたには相違そういありませんが、はっきりとは存じませんでした。中へ這入った
  のは無論今日が初めてでございます。
  さきほど岡田さんが紹介しょうかいかたがたちょっとお話になった通りこの春何か講演をというご注文で
  ありましたが、その当時は何か差支さしつかえがあって、――岡田さんの方が当人の私よりよくご記憶と
  見えてあなたがたにご納得のできるようにただいまご説明がありましたが、とにかくひとまずお断りを
  致いたさなければならん事になりました。しかしただお断りを致すのもあまり失礼と存じまして、
  この次には参りますからという条件をつけ加えておきました。その時念のためこの次はいつごろに
  なりますかと岡田さんに伺うかがいましたら、此年ことしの十月だというお返事であったので、
  心のうちに春から十月までの日数を大体繰くってみて、それだけの時間があればそのうちにどうにか
  できるだろうと思ったものですから、よろしゅうございますとはっきりお受合うけあい申したので
  あります。ところが幸か不幸か病気に罹かかりまして、九月いっぱい床とこについておりますうちに
  お約束やくそくの十月が参りました。十月にはもう臥ふせってはおりませんでしたけれども、何しろ
  ひょろひょろするので講演はちょっとむずかしかったのです。しかしお約束を忘れてはならないの
  ですから、腹の中では、今に何か云いって来られるだろう来られるだろうと思って、内々ないないは
  怖こわがっていました。

  そのうちひょろひょろもついに癒なおってしまったけれども、こちらからは十月末まで何の
  ご沙汰さたもなく打ち過ぎました。私は無論病気の事をご通知はしておきませんでしたが、二三の新聞に
  ちょっと出たという話ですから、あるいはその辺の事情を察せられて、誰だれかが私の代りに講演を
  やって下さったのだろうと推測して安心し出しました。ところへまた岡田さんがまた突然とつぜん
  見えたのであります。岡田さんはわざわざ長靴を穿はいて見えたのであります。
  (もっとも雨の降る日であったからでもありましょうが、)そう云った身拵みごしらえで、
  早稲田わせだの奥おくまで来て下すって、例の講演は十一月の末まで繰り延ばす事にしたから
  約束通りやってもらいたいというご口上なのです。私はもう責任を逃のがれたように考えていたもの
  ですから実は少々驚おどろきました。しかしまだ一カ月も余裕よゆうがあるから、その間にどうか
  なるだろうと思って、よろしゅうございますとまたご返事を致しました。

  % \include{Proposed Method}
  % \include{evaluation}
  % \include{conclusion}

  % Back matter
  \backmatter
  %\appendix
  %\bibliography{foo}

\end{document}
